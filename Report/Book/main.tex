%%%%%%%%%%%%%%%%%%%%%%%%%%%%%%%%%%%%%%%%%%%%%%%%%%
% FORMATING UNTUK BUKU
%%%%%%%%%%%%%%%%%%%%%%%%%%%%%%%%%%%%%%%%%%%%%%%%%%
\documentclass[a4paper,12pt,oneside]{book}
\usepackage{blindtext}
\usepackage{geometry}
\usepackage{titlesec}
\usepackage{hyperref}
\usepackage{afterpage}
\usepackage{indentfirst}
\usepackage{etoolbox}
\usepackage{lmodern}
\usepackage{graphicx}
\usepackage{sectsty}
\graphicspath{ {./} }

\font\tfont=cmr12 at 14pt
\sectionfont{\fontsize{12}{15}\selectfont}

\setlength{\parindent}{2em}
\setlength{\parskip}{0.5em}
\newcommand\blankpage{%
	\null
	\thispagestyle{empty}%
	\addtocounter{page}{-1}%
	\newpage}

\hypersetup{
	colorlinks,
	citecolor=black,
	filecolor=black,
	linkcolor=black,
	urlcolor=black
}
\geometry{
	left=40mm,
	top=30mm,
	right=30mm,
	bottom=30mm,
}
\renewcommand{\chaptername}{BAB}
\renewcommand{\contentsname}{Daftar Isi}
\renewcommand{\figurename}{Gambar}
\renewcommand{\tablename}{Tabel}
\renewcommand{\listfigurename}{Daftar Gambar}
\renewcommand{\listtablename}{Daftar Tabel}

\titleformat{\chapter}[display]
{\normalfont\bfseries}
{BAB \thechapter \centering}{-2ex}
{
	\rule{\textwidth}{0pt}
	\vspace{0ex}
	\centering}
[\vspace{-5ex}\rule{\textwidth}{0pt}]

\titlespacing*{\chapter}{0pt}{-50pt}{14pt}
%%%%%%%%%%%%%%%%%%%%%%%%%%%%%%%%%%%%%%%%%%%%%%%%%%
% BEGIN BOOK
%%%%%%%%%%%%%%%%%%%%%%%%%%%%%%%%%%%%%%%%%%%%%%%%%%
\begin{document}
\title{\tfont DESAIN DAN SIMULASI\\
	PERLINDUNGAN PROPERTI INTELEKTUAL\\
	MENGGUNAKAN ALGORITME FILTER DIGITAL\\.\\	
	DESIGN AND SIMULATION\\
	OF INTELECTUAL PROPERTIES PROTECTION\\
	USING DIGITAL FILTER ALGORITHM
}
\author{
	Disusun sebagai syarat untukmemperoleh gelar Sarjana Teknik\\
	pada Program Studi S1 Sistem Komputer\\
	Universitas Telkom\\\\
	Oleh:\\\\
	\textbf{Hanjara Cahya Adhyatma}\\
	\textbf{1104131113}\\
	\includegraphics[scale=0.13]{logo}
}

\date{
	FAKULTAS TEKNIK ELEKTRO\\
	UNIVERSITAS TELKOM\\
	BANDUNG\\
	2017}
\maketitle
\tableofcontents
\listoffigures
\listoftables
\addtocounter{page}{-3}
\afterpage{\blankpage}
\input{bab1}
\afterpage{\blankpage}
\chapter{Tinjauan Pustaka}
\blindtext[2]

\section{Pekerjaan Sebelumnya dan Keterkaitan}
\blindtext[3]
\section{Perancangan dan Implementasi}
\blindtext[2]

\afterpage{\blankpage}
\chapter{Desain dan Simulasi}
\blindtext[2]

\section{Studi Literatur}
\blindtext[2]
\section{Analisis}
\blindtext[3]
\section{Perancangan}
\blindtext[6]
\section{Pengujian}
\blindtext[3]
\section{Keluaran}
\blindtext[2]
\begin{figure}
	\caption{fig A}
\end{figure}
\begin{figure}
	\caption{fig B}
\end{figure}

\afterpage{\blankpage}
\chapter{Pengujian dan Analisis}
\blindtext[2]
\afterpage{\blankpage}
\chapter{Kesimpulan dan Saran}
\blindtext[2]
\afterpage{\blankpage}
\end{document}
