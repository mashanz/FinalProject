%
% Daftar Pustaka 
% 

% 
% Tambahkan pustaka yang digunakan setelah perintah berikut. 
% 
\begin{thebibliography}{4}

\bibitem{latex.intro}
{Jeff Clark. (n.d). \f{Introduction to LaTeX}.
26 Januari 2010. \url{http://frodo.elon.edu/tutorial/tutorial/node3.html}.}

\bibitem{chapman}
R. Chapman and T. S. Durrani, “IP Protection of DSP Algorithms for System on Chip Implementation,” vol. 48, no. 3, pp. 854–861, 2000.

\bibitem{water}
“Watermarking Techniques for Electronic Circuit Design,” no. 1, pp. 1–17.

\bibitem{lui}
Q. Liu, W. Ji, Q. Chen, and T. Mak, “IP Protection of Mesh NoCs Using Square Spiral Routing,” vol. 24, no. 4, pp. 1560–1573, 2016.

\bibitem{cui}
A. Cui, C. Chang, S. Member, S. Tahar, and S. Member, “A Robust FSM Watermarking Scheme for IP Protection of Sequential Circuit Design,” vol. 30, no. 5, pp. 678–690, 2011.

\bibitem{nie}
T. Nie, “Performance Evaluation for IP Protection Watermarking Techniques.”

\bibitem{zhang1}
J. Zhang, Y. Lin, Y. Lyu, G. Qu, and S. Member, “A PUF-FSM Binding Scheme for FPGA IP Protection and Pay-Per-Device Licensing,” vol. 10, no. 6, pp. 1137–1150, 2015.

\bibitem{zhang2}
J. Zhang, Y. Lin, Q. Wu, and W. Che, “Watermarking FPGA Bitfile for Intellectual Property Protection,” pp. 764–771.

\bibitem{kahng}
A. B. Kahng et al., “Watermarking Techniques for Intellectual Property Protection.”

\bibitem{mosh}
V. G. Moshnyaga and H. Nita, “STG-based Detection of Power Virus Inputs in FSM.”

\end{thebibliography}

