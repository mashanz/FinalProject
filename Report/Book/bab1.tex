\chapter{Pendahuluan}
\section{Latar Belakang}

Integrated Circuit (IC) merupakan modul teknologi dasar dari perangkat elektronika tertanam modern. Dengan berkembangnya teknologi IC yang mengutamakan ukuran kecil, dan performa yang tinggi serta dengan harga yang murah membuat teknologi IC semakin diminati [1].

Dengan ukuran modul yang sangat kecil dan banyaknya komponen
pembangun, kerja sama antara desainer dilakukan untuk membangun sebuah modul VLSI sehingga setiap desainer dapat fokus mendesain salah satu fungsi yang terdapat dalam modul tersebut. Kerja sama dilakukan untuk mempermudah pembuatan desain VLSI yang memiliki tingkat kerumitan yang tinggi. Desainer juga dapat mempercepat waktu mendesain dengan menggunakan kode sumber yang sudah ada atau bekerja sama secara paralel membuat masing-masing modul yang nantinya akan digabung menjadi sebuah modul utama VLSI.

Setelah modul selesai dibuat maka modul siap untuk di-produksi. Dalam proses produksi modul perusahaan tempat desainer bekerja tidak perlu memiliki pabrik produksi modul sendiri, perusahaan dapat bekerja sama dengan mitra percetakan yang akan memproduksi modul buatan perusahaan modul tersebut. Cara kerja sama seperti ini disebut dengan Fabless Manufacturing [2]. Ketika akan memproduksi IC, perusahaan harus menyerahkan blueprint modul VLSI ke percetakan, namun blueprint tersebut tidak terjamin kerahasiaan nya serta memungkinkan plagiarisme desain oleh oknum perusahaan atau pihak ketiga yang tertarik menggunakan desain VLSI yang telah diserahkan untuk di-produksi.

Dengan memberikan rangkaian watermark sebagai pengamanan pada cetak biru VLSI siap cetak yang menandakan kepemilikan dari desainer atau perusahaan produsen modul akan melindungi dari kecurangan pihak lain yang akan mencuri desain. Sehingga kemungkinan pencurian atau plagiarisme yang menyebabkan kerugian pada perusahaan atau desainer karena desain nya dicuri atau di-plagiat berkurang

\section{Rumusan Masalah}
Berikut ini dijelaskan rumusan masalah yang dihadapi dalam proposal penelitian Intelectual Property Protection (IPP) menggunakan metode Digital \textit{Filter Algorithm using Logical Polymorph Gate Key Verification} :
\paragraph{a.} Dengan metode \textit{Fabless Manufacturing}, desain modul yang siap diproduksi diserahkan kepada perusahaan percetakan mitra sehingga mitra dapat mengetahui desain modul dari desainer yang memungkinkan desain dapat dicuri oleh oknum percetakan atau pihak ketiga yang tertarik dengan desain tersebut.
\paragraph{b.}Desain modul rawan terhadap plagiarisme karena desain elektronik sangat mudah ditiru, sehingga pengamanan desain harus dilakukan agar desain tidak mudah untuk dicuri atau di-plagiat.
\paragraph{c.}Apabila pihak ketiga mencuri desain, desainer dapat mengklaim modul tersebut dengan bukti dari pengamanan watermark yang telah tertanam dalam IC menggunakan teknik pemanggilan watermark yang hanya diketahui oleh desainer yang mendesain IC tersebut.

\section{Tujuan}
Berikut merupakan tujuan pengamanan desain modul yang siap cetak sehingga aman terhadap pencurian hak cipta :
\paragraph{a.} Merancang rangkaian pengamanan dalam sebuah chip design sebagai bukti kepemilikan desain (ownership) atau watermarking.
\paragraph{b.} Desain chip yang telah diberi rangkaian watermark akan dianalisis perubahan performa dari desain sebelum dan sesudah watermarking serta kemungkinan watermark di-modifikasi oleh pihak lain atau reverse engineering untuk digunakan kembali oleh pengguna yang tidak sah.
\paragraph{c.} Rangkaian ini akan ditanam di dalam chip yang pemanggilan informasi pemilik dari chip hanya diketahui oleh pemilik cipta.

\section{Batasan Masalah}
Dalam penelitian ini rancangan desain VLSI yang disisipkan watermark membatasi masalah serta pembahasan yang akan diteliti sebagai berikut :
\paragraph{a.} Tidak membuat modul IC VLSI spesifik, namun menggunakan yang sudah ada dan menyisipkan dengan watermark.
\paragraph{b.} Menyisipkan rangkaian dengan data watermark dan tidak membahas detail data dari pemilik cipta.
\paragraph{c.} Watermarking yang dilakukan untuk satu chip IC dan tidak mewatermark masing-masing modul yang ter-integrasi dalam chip IC.

\section{Hipotesis}
\blindtext[1]
\begin{table}
	\caption{table A}
\end{table}
\begin{table}
	\caption{table B}
\end{table}